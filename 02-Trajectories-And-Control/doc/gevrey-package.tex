\usepackage{amsfonts,amsthm}
\usepackage{framed}
\usepackage{makeidx}
\usepackage{floatflt}
\usepackage{subfigure}
\usepackage{lscape}
\hyphenation{con-di-tion}
\numberwithin{equation}{section}

% Partielle Ableitung  #1: Zaehler, #2: Nenner
\newcommand{\pdiff}[2]
{
 \frac{\partial #1}{\partial #2}
}

\newtheorem{lmm}{Lemma}[section]
\newtheorem{thrm}[lmm]{Theorem}
\newtheorem{note}[lmm]{Note}

\newcommand{\pAbl}[1]{\frac{\partial}{\partial #1}} % partielle Ableitung
% Einige Wertebereiche
\newcommand{\domC}{\mathbb{C}} %Natuerliche Zahlen
\newcommand{\domN}{\mathbb{N}} %Natuerliche Zahlen
\newcommand{\domR}{\mathbb{R}} %Reelle Zahle
\def\RR{\mathbb R}
\newcommand{\domRp}{\mathbb{R}^+} %Positive Reele Zahlen
\newcommand{\Gras}[1]{{#1}}%\boldmath
\newcommand{\bolpi}{{\Gras \pi}}
\newcommand{\bols}{{\Gras s}}
\newcommand{\Deltas}{{\Gras \delta}}
\newcommand{\boltheta}{{\Gras \theta}}
\newcommand{\bolsigma}{{\Gras \sigma}}
\newcommand{\bolbeta}{{\Gras \beta}}
\newcommand{\bolepsilon}{{\Gras \epsilon}}
 \newcommand{\ie}{{i.e.},\ }
 \newcommand{\cf}{{cf.}\ }
\newcommand{\quoted}[1]{{\glqq #1\grqq}}
\newcommand{\myparagraph}[1]{\underline{\textbf{#1}}\\[0ex]}
\newcounter{exercise}%[chapter]
\numberwithin{exercise}{section}
\newenvironment{mycompexer}{%
\refstepcounter{exercise}%
\pagebreak[3]
\begin{shadedwithicon}{13.5}{\protect{\parbox{1cm}{\center\includegraphics[width=0.75cm]{ico_compexer}}}\nopagebreak}%
{\noindent\underline{\textbf{Exercise \theexercise:}}\\[0.5ex]}%
}{\end{shadedwithicon}}
\newcommand{\df}{\mathrm{d}}

%%% Local Variables: 
%%% mode: latex
%%% TeX-master: t
%%% End: 
